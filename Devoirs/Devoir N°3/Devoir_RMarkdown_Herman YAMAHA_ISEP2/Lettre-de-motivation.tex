% Options for packages loaded elsewhere
\PassOptionsToPackage{unicode}{hyperref}
\PassOptionsToPackage{hyphens}{url}
%
\documentclass[
  13pt,
]{article}
\usepackage{amsmath,amssymb}
\usepackage{iftex}
\ifPDFTeX
  \usepackage[T1]{fontenc}
  \usepackage[utf8]{inputenc}
  \usepackage{textcomp} % provide euro and other symbols
\else % if luatex or xetex
  \usepackage{unicode-math} % this also loads fontspec
  \defaultfontfeatures{Scale=MatchLowercase}
  \defaultfontfeatures[\rmfamily]{Ligatures=TeX,Scale=1}
\fi
\usepackage{lmodern}
\ifPDFTeX\else
  % xetex/luatex font selection
    \setmainfont[]{Times New Roman}
\fi
% Use upquote if available, for straight quotes in verbatim environments
\IfFileExists{upquote.sty}{\usepackage{upquote}}{}
\IfFileExists{microtype.sty}{% use microtype if available
  \usepackage[]{microtype}
  \UseMicrotypeSet[protrusion]{basicmath} % disable protrusion for tt fonts
}{}
\makeatletter
\@ifundefined{KOMAClassName}{% if non-KOMA class
  \IfFileExists{parskip.sty}{%
    \usepackage{parskip}
  }{% else
    \setlength{\parindent}{0pt}
    \setlength{\parskip}{6pt plus 2pt minus 1pt}}
}{% if KOMA class
  \KOMAoptions{parskip=half}}
\makeatother
\usepackage{xcolor}
\usepackage[top=2.5cm,bottom=2.5cm,left=3cm,right=2.5cm,margin=1in]{geometry}
\usepackage{graphicx}
\makeatletter
\def\maxwidth{\ifdim\Gin@nat@width>\linewidth\linewidth\else\Gin@nat@width\fi}
\def\maxheight{\ifdim\Gin@nat@height>\textheight\textheight\else\Gin@nat@height\fi}
\makeatother
% Scale images if necessary, so that they will not overflow the page
% margins by default, and it is still possible to overwrite the defaults
% using explicit options in \includegraphics[width, height, ...]{}
\setkeys{Gin}{width=\maxwidth,height=\maxheight,keepaspectratio}
% Set default figure placement to htbp
\makeatletter
\def\fps@figure{htbp}
\makeatother
\setlength{\emergencystretch}{3em} % prevent overfull lines
\providecommand{\tightlist}{%
  \setlength{\itemsep}{0pt}\setlength{\parskip}{0pt}}
\setcounter{secnumdepth}{-\maxdimen} % remove section numbering
\ifLuaTeX
  \usepackage{selnolig}  % disable illegal ligatures
\fi
\usepackage{bookmark}
\IfFileExists{xurl.sty}{\usepackage{xurl}}{} % add URL line breaks if available
\urlstyle{same}
\hypersetup{
  pdftitle={Lettre de motivation},
  pdfauthor={Herman YAMAHA},
  hidelinks,
  pdfcreator={LaTeX via pandoc}}

\title{Lettre de motivation}
\author{Herman YAMAHA}
\date{02 mai 2025}

\begin{document}
\maketitle

Herman YAMAHA\\
Étudiant en 2ᵉ année à l'ENSAE de Dakar\\
Téléphone : +221 77 552 54 41\\
Email :
\href{mailto:herman.yamaha@gmail.com}{\nolinkurl{herman.yamaha@gmail.com}}

\begin{center}\rule{0.5\linewidth}{0.5pt}\end{center}

\begin{flushright}
Dakar, le 02 mai 2025  
\end{flushright}

\begin{center}\rule{0.5\linewidth}{0.5pt}\end{center}

\begin{flushright}
A l'attention du directeur général \\
              ANSD \\
   Cerf-Volant, Colobane \\
       Dakar, Sénégal
\end{flushright}

\begin{center}\rule{0.5\linewidth}{0.5pt}\end{center}

Objet : Candidature à un stage en économie appliquée -- Étudiant à
l'ENSAE de Dakar

Monsieur le directeur général,

Actuellement étudiant en deuxième année à l'École Nationale de la
Statistique et de l'Analyse Économique (ENSAE) de Dakar, je suis à la
recherche d'un stage académique d'une durée de deux à trois mois à
partir de juillet. Intéressé par l'analyse économique et statistique
appliquée aux politiques publiques, je souhaite intégrer votre structure
afin d'y renforcer mes compétences et contribuer activement à vos
travaux.

Ma formation à l'ENSAE m'a permis d'acquérir des bases solides en
microéconomie, économétrie, statistique inférentielle ainsi qu'en
programmation avec R et Python. En parallèle, j'ai mené des travaux
pratiques sur l'analyse de données agricoles selon les recommandations
de la Banque Mondiale, que j'ai publiés sur GitHub. Ces expériences
m'ont donné le goût du travail rigoureux et m'ont familiarisé avec les
outils de traitement des données réelles.

Rigoureux, curieux et doté d'un bon esprit d'analyse, je suis motivé à
m'investir pleinement dans les missions que vous voudrez bien me
confier. Intégrer votre équipe serait pour moi l'occasion de découvrir
un environnement professionnel exigeant, tout en contribuant
concrètement à vos projets.

Je vous remercie par avance de l'attention que vous porterez à ma
candidature et me tiens à votre disposition pour tout entretien à votre
convenance.

\begin{flushright}
L’intéressé\\
Herman YAMAHA
\end{flushright}

\end{document}
